% !TeX encoding = UTF-8
% !TeX program = xelatex
% !TeX spellcheck = en_US

\documentclass[degree=postdoc]{thuthesis}
  % 学位 degree:
  %   doctor | master | bachelor | postdoc
  % 学位类型 degree-type:
  %   academic(默认)| professional
  % 语言 language
  %   chinese(默认)| english
  % 字体库 fontset
  %   windows | mac | fandol | ubuntu
  % 建议终版使用 Windows 平台的字体编译


% 论文基本配置,加载宏包等全局配置
% !TeX root = ./thuthesis-example.tex

% 论文基本信息配置

\thusetup{
  %******************************
  % 注意:
  %   1. 配置里面不要出现空行
  %   2. 不需要的配置信息可以删除
  %   3. 建议先阅读文档中所有关于选项的说明
  %******************************
  %
  % 输出格式
  %   选择打印版(print)或用于提交的电子版(electronic),前者会插入空白页以便直接双面打印
  %
  output = print,
  %
  % 标题
  %   可使用“\\”命令手动控制换行
  %
  title  = {清华大学学位论文 \LaTeX{} 模板\\使用示例文档 v\version},
  title* = {An Introduction to \LaTeX{} Thesis Template of Tsinghua
            University v\version},
  %
  % 学科门类
  %   1. 学术型
  %      - 中文
  %        需注明所属的学科门类,例如:
  %        哲学、经济学、法学、教育学、文学、历史学、理学、工学、农学、医学、
  %        军事学、管理学、艺术学
  %      - 英文
  %        博士:Doctor of Philosophy
  %        硕士:
  %          哲学、文学、历史学、法学、教育学、艺术学门类,公共管理学科
  %          填写“Master of Arts“,其它填写“Master of Science”
  %   2. 专业型
  %      直接填写专业学位的名称,例如:
  %      教育博士、工程硕士等
  %      Doctor of Education, Master of Engineering
  %   3. 本科生不需要填写
  %
  degree-category  = {工学硕士},
  degree-category* = {Master of Science},
  %
  % 培养单位
  %   填写所属院系的全名
  %
  department = {计算机科学与技术系},
  %
  % 学科
  %   1. 研究生学术型学位,获得一级学科授权的学科填写一级学科名称,其他填写二级学科名称
  %   2. 本科生填写专业名称,第二学位论文需标注“(第二学位)”
  %
  discipline  = {计算机科学与技术},
  discipline* = {Computer Science and Technology},
  %
  % 专业领域
  %   1. 设置专业领域的专业学位类别,填写相应专业领域名称
  %   2. 2019 级及之前工程硕士学位论文,在 `engineering-field` 填写相应工程领域名称
  %   3. 其他专业学位类别的学位论文无需此信息
  %
  % professional-field  = {计算机技术},
  % professional-field* = {Computer Technology},
  %
  % 姓名
  %
  author  = {薛瑞尼},
  author* = {Xue Ruini},
  %
  % 指导教师
  %   中文姓名和职称之间以英文逗号“,”分开,下同
  %
  supervisor  = {郑纬民, 教授},
  supervisor* = {Professor Zheng Weimin},
  %
  % 副指导教师
  %
  associate-supervisor  = {陈文光, 教授},
  associate-supervisor* = {Professor Chen Wenguang},
  %
  % 联合指导教师
  %
  % co-supervisor  = {某某某, 教授},
  % co-supervisor* = {Professor Mou Moumou},
  %
  % 日期
  %   使用 ISO 格式;默认为当前时间
  %
  % date = {2019-07-07},
  %
  % 是否在中文封面后的空白页生成书脊(默认 false)
  %
  include-spine = false,
  %
  % 密级和年限
  %   秘密, 机密, 绝密
  %
  % secret-level = {秘密},
  % secret-year  = {10},
  %
  % 博士后专有部分
  %
  % clc                = {分类号},
  % udc                = {UDC},
  % id                 = {编号},
  % discipline-level-1 = {计算机科学与技术},  % 流动站(一级学科)名称
  % discipline-level-2 = {系统结构},          % 专业(二级学科)名称
  % start-date         = {2011-07-01},        % 研究工作起始时间
}

% 载入所需的宏包

% 定理类环境宏包
\usepackage{amsthm}
% 也可以使用 ntheorem
% \usepackage[amsmath,thmmarks,hyperref]{ntheorem}

\thusetup{
  %
  % 数学字体
  % math-style = GB,  % GB | ISO | TeX
  math-font  = xits,  % stix | xits | libertinus
}

% 可以使用 nomencl 生成符号和缩略语说明
% \usepackage{nomencl}
% \makenomenclature

% 表格加脚注
\usepackage{threeparttable}

% 表格中支持跨行
\usepackage{multirow}

% 固定宽度的表格。
% \usepackage{tabularx}

% 跨页表格
\usepackage{longtable}

% 算法
\usepackage{algorithm}
\usepackage{algorithmic}

% 量和单位
\usepackage{siunitx}

% 参考文献使用 BibTeX + natbib 宏包
% 顺序编码制
\usepackage[sort]{natbib}
\bibliographystyle{thuthesis-numeric}

% 著者-出版年制
% \usepackage{natbib}
% \bibliographystyle{thuthesis-author-year}

% 本科生参考文献的著录格式
% \usepackage[sort]{natbib}
% \bibliographystyle{thuthesis-bachelor}

% 参考文献使用 BibLaTeX 宏包
% \usepackage[style=thuthesis-numeric]{biblatex}
% \usepackage[style=thuthesis-author-year]{biblatex}
% \usepackage[style=apa]{biblatex}
% \usepackage[style=mla-new]{biblatex}
% 声明 BibLaTeX 的数据库
% \addbibresource{ref/refs.bib}

% 定义所有的图片文件在 figures 子目录下
\graphicspath{{figures/}}

% 数学命令
\makeatletter
\newcommand\dif{%  % 微分符号
  \mathop{}\!%
  \ifthu@math@style@TeX
    d%
  \else
    \mathrm{d}%
  \fi
}
\makeatother

% hyperref 宏包在最后调用
\usepackage{hyperref}



\begin{document}

% 封面
\maketitle

% 学位论文指导小组、公开评阅人和答辩委员会名单
% 本科生不需要
% !TeX root = ../thuthesis-example.tex

\begin{committee}[name={学位论文指导小组、公开评阅人和答辩委员会名单}]

  \newcolumntype{C}[1]{@{}>{\centering\arraybackslash}p{#1}}

  \section*{指导小组名单}

  \begin{center}
    \begin{tabular}{C{3cm}C{3cm}C{9cm}@{}}
      李XX & 教授     & 清华大学 \\
      王XX & 副教授   & 清华大学 \\
      张XX & 助理教授 & 清华大学 \\
    \end{tabular}
  \end{center}


  \section*{公开评阅人名单}

  \begin{center}
    \begin{tabular}{C{3cm}C{3cm}C{9cm}@{}}
      刘XX & 教授   & 清华大学                    \\
      陈XX & 副教授 & XXXX大学                    \\
      杨XX & 研究员 & 中国XXXX科学院XXXXXXX研究所 \\
    \end{tabular}
  \end{center}


  \section*{答辩委员会名单}

  \begin{center}
    \begin{tabular}{C{2.75cm}C{2.98cm}C{4.63cm}C{4.63cm}@{}}
      主席 & 赵XX                  & 教授                    & 清华大学       \\
      委员 & 刘XX                  & 教授                    & 清华大学       \\
          & \multirow{2}{*}{杨XX} & \multirow{2}{*}{研究员} & 中国XXXX科学院 \\
          &                       &                         & XXXXXXX研究所  \\
          & 黄XX                  & 教授                    & XXXX大学       \\
          & 周XX                  & 副教授                  & XXXX大学       \\
      秘书 & 吴XX                  & 助理研究员              & 清华大学       \\
    \end{tabular}
  \end{center}

\end{committee}



% 也可以导入 Word 版转的 PDF 文件
% \begin{committee}[file=figures/committee.pdf]
% \end{committee}


% 使用授权的说明
\copyrightpage
% 将签字扫描后授权文件 scan-copyright.pdf 替换原始页面
% \copyrightpage[file=scan-copyright.pdf]

\frontmatter
% !TeX root = ../thuthesis-example.tex

% 中英文摘要和关键字

\begin{abstract}
拓扑优化利用计算手段优化材料分布,旨在获得预期性能并满足特定约束的结构,是现代智能制造领域的重要设计方法。然而,拓扑优化过程中产生的结构通常具有非常复杂的几何与拓扑特征,给结构的表示、分析和优化都带来了极大的挑战。本工作利用隐式表示灵活性和可微分性的特点,针对不同的结构设计目标和约束,以提高算法效率和应用可扩展性为目的,开展了以下创新性研究工作:
  
(1)传统显式方法在设计多样化、轻量化和物理可靠的薄壳结构时仍存在诸多挑战,针对于次,本文通过开发一种新颖的隐式设计框架,实现了在薄壳结构上的高效雕刻设计。基于参数化表示的雕刻模板(包括规则、非规则和个性化三种),本文通过优化模板尺寸和方向等参数,达到在指定材料消耗下的结构刚度最大化。本文的方法通过完全的隐式函数操作完成结构的表示、分析和优化,避免了传统有限元方法中繁琐耗时的重新网格化步骤,极大地提高了薄壳结构的设计效率。

(2)空心化通过从三维模型内部体积中去除材料来实现轻量化,同时还可保持可行的机械性能。然而,空心化模型在增材制造过程中往往需要额外的支撑材料,大大抵消了轻量化效果。本文提出了两种创新方法以解决该问题:第一种使用基于椭球体的连续函数表示,设计和优化自然具有自支撑性的空心化结构,本文提出了一种高效的优化策略,以确定椭球空腔的形状、位置和拓扑结构,旨在实现最小化材料成本、最大化结构刚度和确保自支撑性等多种目标;第二种通过一种高效的可微分通道设计框架,以连通空心腔体使其能够顺利导出打印过程中残留在腔体内的制造材料。

(3)拓扑优化较高的计算复杂性和周期较长的迭代过程严重影响了效率,这给其实际应用都带来了严重阻碍。为了解决这一挑战,本文提出了创新框架IF-TONIR,利用神经网络隐式表示复杂结构形状,并基于数据驱动的方式实现端到端的拓扑优化,能够从不同设计域的不同边界条件直接预测最优化结构。此外,本文提出了基于持续同调技术的拓扑损失函数来训练网络模型,有效惩罚了预测结构中断裂的存在,从而提高了生成结构的整体物理可靠性。

本文研究提出的基于隐式方法的复杂结构表示、分析和优化方法,具有计算效率高、设计效率高和适用性强等优点,有望在工程设计领域得到广泛应用,推动相关领域的技术进步。

  % 关键词用“英文逗号”分隔,输出时会自动处理为正确的分隔符
  \thusetup{
    keywords = {拓扑优化, 隐式表示, 变分自编码器, 薄壳结构, 自支撑结构},
  }
\end{abstract}

\begin{abstract*}
Topology optimization utilizes computational methods to optimize material distribution, aiming to obtain structures with expected performance and specific constraints. It is an important design method in the field of modern intelligent manufacturing. However, the structures generated during the topology optimization process often have very complex geometric and topological features, posing great challenges to the representation, analysis, and optimization of the structures. This work leverages the flexibility and differentiability of implicit representations to address different structural design objectives and constraints, with the goal of improving algorithmic efficiency and application scalability. The following innovative research work has been carried out:

(1) Traditional explicit methods still face many challenges when designing diverse, lightweight, and physically reliable thin-shell structures. To address this, this paper develops a novel implicit design framework to achieve efficient carving design on thin-shell structures. Based on parameterized carving templates (including regular, irregular, and personalized types), this paper optimizes template size, orientation, and other parameters to maximize structural stiffness under specified material consumption. The method in this paper completes the representation, analysis, and optimization of the structure through fully implicit function operations, avoiding the tedious and time-consuming remeshing steps in traditional finite element methods, greatly improving the design efficiency of thin-shell structures.

(2) Hollowing achieves lightweight design by removing material from the internal volume of three-dimensional models while maintaining feasible mechanical performance. However, hollowed models often require additional support materials during the additive manufacturing process, greatly offsetting the lightweight effect. This paper proposes two innovative methods to solve this problem: The first uses a continuous function representation based on ellipsoids to design and optimize naturally self-supporting hollowed structures. An efficient optimization strategy is proposed to determine the shape, position, and topological structure of ellipsoidal cavities, aiming to achieve multiple objectives such as minimizing material cost, maximizing structural stiffness, and ensuring self-supportability. The second method uses an efficient differentiable channel design framework to connect hollow cavities, allowing for smooth extraction of manufacturing materials remaining inside the cavities during the printing process.

(3) The high computational complexity and long iterative process of topology optimization seriously affect efficiency, posing significant obstacles to its practical application. To address this challenge, this paper proposes an innovative framework, IF-TONIR, which utilizes neural networks to implicitly represent complex structural shapes and achieves end-to-end topology optimization based on a data-driven approach, enabling direct prediction of optimized structures from different design domains and boundary conditions. Furthermore, this paper proposes a topology loss function based on persistent homology techniques to train the network model, effectively penalizing the existence of discontinuities in the predicted structures, thus improving the overall physical reliability of the generated structures.

The complex structure representation, analysis, and optimization methods based on implicit approaches proposed in this research have advantages such as high computational efficiency, high design efficiency, and strong applicability. They are expected to be widely applied in the field of engineering design, promoting technological progress in related fields.

  % Use comma as separator when inputting
  \thusetup{
    keywords* = {Topology Optimization, Implicit Representations, Variational Auto-Encoder, Thin-Shell Structures, Self-Supporting Structures},
  }
\end{abstract*}


% 目录
\tableofcontents

% 插图和附表清单
% 本科生的插图索引和表格索引需要移至正文之后、参考文献前
% \listoffiguresandtables  % 插图和附表清单(仅限研究生)
\listoffigures           % 插图清单
\listoftables            % 附表清单

% 符号对照表
% !TeX root = ../thuthesis-example.tex

\begin{denotation}[3cm]
  \item[PI] 聚酰亚胺
  \item[MPI] 聚酰亚胺模型化合物,N-苯基邻苯酰亚胺
  \item[PBI] 聚苯并咪唑
  \item[MPBI] 聚苯并咪唑模型化合物,N-苯基苯并咪唑
  \item[PY] 聚吡咙
  \item[PMDA-BDA] 均苯四酸二酐与联苯四胺合成的聚吡咙薄膜
  \item[MPY] 聚吡咙模型化合物
  \item[As-PPT] 聚苯基不对称三嗪
  \item[MAsPPT] 聚苯基不对称三嗪单模型化合物,3,5,6-三苯基-1,2,4-三嗪
  \item[DMAsPPT] 聚苯基不对称三嗪双模型化合物(水解实验模型化合物)
  \item[S-PPT] 聚苯基对称三嗪
  \item[MSPPT] 聚苯基对称三嗪模型化合物,2,4,6-三苯基-1,3,5-三嗪
  \item[PPQ] 聚苯基喹噁啉
  \item[MPPQ] 聚苯基喹噁啉模型化合物,3,4-二苯基苯并二嗪
  \item[HMPI] 聚酰亚胺模型化合物的质子化产物
  \item[HMPY] 聚吡咙模型化合物的质子化产物
  \item[HMPBI] 聚苯并咪唑模型化合物的质子化产物
  \item[HMAsPPT] 聚苯基不对称三嗪模型化合物的质子化产物
  \item[HMSPPT] 聚苯基对称三嗪模型化合物的质子化产物
  \item[HMPPQ] 聚苯基喹噁啉模型化合物的质子化产物
  \item[PDT] 热分解温度
  \item[HPLC] 高效液相色谱(High Performance Liquid Chromatography)
  \item[HPCE] 高效毛细管电泳色谱(High Performance Capillary lectrophoresis)
  \item[LC-MS] 液相色谱-质谱联用(Liquid chromatography-Mass Spectrum)
  \item[TIC] 总离子浓度(Total Ion Content)
  \item[\textit{ab initio}] 基于第一原理的量子化学计算方法,常称从头算法
  \item[DFT] 密度泛函理论(Density Functional Theory)
  \item[$E_a$] 化学反应的活化能(Activation Energy)
  \item[ZPE] 零点振动能(Zero Vibration Energy)
  \item[PES] 势能面(Potential Energy Surface)
  \item[TS] 过渡态(Transition State)
  \item[TST] 过渡态理论(Transition State Theory)
  \item[$\increment G^\neq$] 活化自由能(Activation Free Energy)
  \item[$\kappa$] 传输系数(Transmission Coefficient)
  \item[IRC] 内禀反应坐标(Intrinsic Reaction Coordinates)
  \item[$\nu_i$] 虚频(Imaginary Frequency)
  \item[ONIOM] 分层算法(Our own N-layered Integrated molecular Orbital and molecular Mechanics)
  \item[SCF] 自洽场(Self-Consistent Field)
  \item[SCRF] 自洽反应场(Self-Consistent Reaction Field)
\end{denotation}



% 也可以使用 nomencl 宏包,需要在导言区
% \usepackage{nomencl}
% \makenomenclature

% 在这里输出符号说明
% \printnomenclature[3cm]

% 在正文中的任意为都可以标题
% \nomenclature{PI}{聚酰亚胺}
% \nomenclature{MPI}{聚酰亚胺模型化合物,N-苯基邻苯酰亚胺}
% \nomenclature{PBI}{聚苯并咪唑}
% \nomenclature{MPBI}{聚苯并咪唑模型化合物,N-苯基苯并咪唑}
% \nomenclature{PY}{聚吡咙}
% \nomenclature{PMDA-BDA}{均苯四酸二酐与联苯四胺合成的聚吡咙薄膜}
% \nomenclature{MPY}{聚吡咙模型化合物}
% \nomenclature{As-PPT}{聚苯基不对称三嗪}
% \nomenclature{MAsPPT}{聚苯基不对称三嗪单模型化合物,3,5,6-三苯基-1,2,4-三嗪}
% \nomenclature{DMAsPPT}{聚苯基不对称三嗪双模型化合物(水解实验模型化合物)}
% \nomenclature{S-PPT}{聚苯基对称三嗪}
% \nomenclature{MSPPT}{聚苯基对称三嗪模型化合物,2,4,6-三苯基-1,3,5-三嗪}
% \nomenclature{PPQ}{聚苯基喹噁啉}
% \nomenclature{MPPQ}{聚苯基喹噁啉模型化合物,3,4-二苯基苯并二嗪}
% \nomenclature{HMPI}{聚酰亚胺模型化合物的质子化产物}
% \nomenclature{HMPY}{聚吡咙模型化合物的质子化产物}
% \nomenclature{HMPBI}{聚苯并咪唑模型化合物的质子化产物}
% \nomenclature{HMAsPPT}{聚苯基不对称三嗪模型化合物的质子化产物}
% \nomenclature{HMSPPT}{聚苯基对称三嗪模型化合物的质子化产物}
% \nomenclature{HMPPQ}{聚苯基喹噁啉模型化合物的质子化产物}
% \nomenclature{PDT}{热分解温度}
% \nomenclature{HPLC}{高效液相色谱(High Performance Liquid Chromatography)}
% \nomenclature{HPCE}{高效毛细管电泳色谱(High Performance Capillary lectrophoresis)}
% \nomenclature{LC-MS}{液相色谱-质谱联用(Liquid chromatography-Mass Spectrum)}
% \nomenclature{TIC}{总离子浓度(Total Ion Content)}
% \nomenclature{\textit{ab initio}}{基于第一原理的量子化学计算方法,常称从头算法}
% \nomenclature{DFT}{密度泛函理论(Density Functional Theory)}
% \nomenclature{$E_a$}{化学反应的活化能(Activation Energy)}
% \nomenclature{ZPE}{零点振动能(Zero Vibration Energy)}
% \nomenclature{PES}{势能面(Potential Energy Surface)}
% \nomenclature{TS}{过渡态(Transition State)}
% \nomenclature{TST}{过渡态理论(Transition State Theory)}
% \nomenclature{$\increment G^\neq$}{活化自由能(Activation Free Energy)}
% \nomenclature{$\kappa$}{传输系数(Transmission Coefficient)}
% \nomenclature{IRC}{内禀反应坐标(Intrinsic Reaction Coordinates)}
% \nomenclature{$\nu_i$}{虚频(Imaginary Frequency)}
% \nomenclature{ONIOM}{分层算法(Our own N-layered Integrated molecular Orbital and molecular Mechanics)}
% \nomenclature{SCF}{自洽场(Self-Consistent Field)}
% \nomenclature{SCRF}{自洽反应场(Self-Consistent Reaction Field)}



% 正文部分
\mainmatter
% !TeX root = ../thuthesis-example.tex

\chapter{论文主要部分的写法}

研究生学位论文撰写,除表达形式上需要符合一定的格式要求外,内容方面上也要遵循一些共性原则。

通常研究生学位论文只能有一个主题(不能是几块工作拼凑在一起),该主题应针对某学科领域中的一个具体问题展开深入、系统的研究,并得出有价值的研究结论。
学位论文的研究主题切忌过大,例如,“中国国有企业改制问题研究”这样的研究主题过大,因为“国企改制”涉及的问题范围太广,很难在一本研究生学位论文中完全研究透彻。



\section{论文的语言及表述}

除国际研究生外,学位论文一律须用汉语书写。
学位论文应当用规范汉字进行撰写,除古汉语研究中涉及的古文字和参考文献中引用的外文文献之外,均采用简体汉字撰写。

国际研究生一般应以中文或英文书写学位论文,格式要求同上。
论文须用中文封面。

研究生学位论文是学术作品,因此其表述要严谨简明,重点突出,专业常识应简写或不写,做到立论正确、数据可靠、说明透彻、推理严谨、文字凝练、层次分明,避免使用文学性质的或带感情色彩的非学术性语言。

论文中如出现一个非通用性的新名词、新术语或新概念,需随即解释清楚。



\section{论文题目的写法}

论文题目应简明扼要地反映论文工作的主要内容,力求精炼、准确,切忌笼统。
论文题目是对研究对象的准确、具体描述,一般要在一定程度上体现研究结论,因此,论文题目不仅应告诉读者这本论文研究了什么问题,更要告诉读者这个研究得出的结论。
例如:“在事实与虚构之间:梅乐、卡彭特、沃尔夫的新闻观”就比“三个美国作家的新闻观研究”更专业、更准确。



\section{摘要的写法}

论文摘要是对论文研究内容的高度概括,应具有独立性和自含性,即应是 一篇简短但意义完整的文章。
通过阅读论文摘要,读者应该能够对论文的研究 方法及结论有一个整体性的了解,因此摘要的写法应力求精确简明。
论文摘要 应包括对问题及研究目的的描述、对使用的方法和研究过程进行的简要介绍、 对研究结论的高度凝练等,重点是结果和结论。

论文摘要切忌写成全文的提纲,尤其要避免“第 1 章……;第 2 章……;……”这样的陈述方式。



\section{引言的写法}

一篇学位论文的引言大致包含如下几个部分:
1、问题的提出;
2、选题背 景及意义;
3、文献综述;
4、研究方法;
5、论文结构安排。
\begin{itemize}
  \item 问题的提出:要清晰地阐述所要研究的问题“是什么”。
    \footnote{选题时切记要有“问题意识”,不要选不是问题的问题来研究。}
  \item 选题背景及意义:论述清楚为什么选择这个题目来研究,即阐述该研究对学科发展的贡献、对国计民生的理论与现实意义等。
  \item 文献综述:对本研究主题范围内的文献进行详尽的综合述评,“述”的同时一定要有“评”,指出现有研究状态,仍存在哪些尚待解决的问题,讲出自己的研究有哪些探索性内容。
  \item 研究方法:讲清论文所使用的学术研究方法。
  \item 论文结构安排:介绍本论文的写作结构安排。
\end{itemize}



\section{正文的写法}

本部分是论文作者的研究内容,不能将他人研究成果不加区分地掺和进来。
已经在引言的文献综述部分讲过的内容,这里不需要再重复。
各章之间要存在有机联系,符合逻辑顺序。



\section{结论的写法}

结论是对论文主要研究结果、论点的提炼与概括,应精炼、准确、完整,使读者看后能全面了解论文的意义、目的和工作内容。
结论是最终的、总体的结论,不是正文各章小结的简单重复。
结论应包括论文的核心观点,主要阐述作者的创造性工作及所取得的研究成果在本领域中的地位、作用和意义,交代研究工作的局限,提出未来工作的意见或建议。
同时,要严格区分自己取得的成果与指导教师及他人的学术成果。

在评价自己的研究工作成果时,要实事求是,除非有足够的证据表明自己的研究是“首次”、“领先”、“填补空白”的,否则应避免使用这些或类似词语。

% !TeX root = ../thuthesis-example.tex

\chapter{图表示例}

\section{插图}

图片通常在 \env{figure} 环境中使用 \cs{includegraphics} 插入,如图~\ref{fig:example} 的源代码。
建议矢量图片使用 PDF 格式,比如数据可视化的绘图;
照片应使用 JPG 格式;
其他的栅格图应使用无损的 PNG 格式。
注意,LaTeX 不支持 TIFF 格式;EPS 格式已经过时。

\begin{figure}
  \centering
  \includegraphics[width=0.5\linewidth]{example-image-a.pdf}
  \caption*{国外的期刊习惯将图表的标题和说明文字写成一段,需要改写为标题只含图表的名称,其他说明文字以注释方式写在图表下方,或者写在正文中。}
  \caption{示例图片标题}
  \label{fig:example}
\end{figure}

若图或表中有附注,采用英文小写字母顺序编号,附注写在图或表的下方。
国外的期刊习惯将图表的标题和说明文字写成一段,需要改写为标题只含图表的名称,其他说明文字以注释方式写在图表下方,或者写在正文中。

如果一个图由两个或两个以上分图组成时,各分图分别以 (a)、(b)、(c)...... 作为图序,并须有分图题。
推荐使用 \pkg{subcaption} 宏包来处理, 比如图~\ref{fig:subfig-a} 和图~\ref{fig:subfig-b}。

\begin{figure}
  \centering
  \subcaptionbox{分图 A\label{fig:subfig-a}}
    {\includegraphics[width=0.35\linewidth]{example-image-a.pdf}}
  \subcaptionbox{分图 B\label{fig:subfig-b}}
    {\includegraphics[width=0.35\linewidth]{example-image-b.pdf}}
  \caption{多个分图的示例}
  \label{fig:multi-image}
\end{figure}



\section{表格}

表应具有自明性。为使表格简洁易读,尽可能采用三线表,如表~\ref{tab:three-line}。
三条线可以使用 \pkg{booktabs} 宏包提供的命令生成。

\begin{table}
  \centering
  \caption{三线表示例}
  \begin{tabular}{ll}
    \toprule
    文件名          & 描述                         \\
    \midrule
    thuthesis.dtx   & 模板的源文件,包括文档和注释 \\
    thuthesis.cls   & 模板文件                     \\
    thuthesis-*.bst & BibTeX 参考文献表样式文件    \\
    \bottomrule
  \end{tabular}
  \label{tab:three-line}
\end{table}

表格如果有附注,尤其是需要在表格中进行标注时,可以使用 \pkg{threeparttable} 宏包。
研究生要求使用英文小写字母 a、b、c……顺序编号,本科生使用圈码 ①、②、③……编号。

\begin{table}
  \centering
  \begin{threeparttable}[c]
    \caption{带附注的表格示例}
    \label{tab:three-part-table}
    \begin{tabular}{ll}
      \toprule
      文件名                 & 描述                         \\
      \midrule
      thuthesis.dtx\tnote{a} & 模板的源文件,包括文档和注释 \\
      thuthesis.cls\tnote{b} & 模板文件                     \\
      thuthesis-*.bst        & BibTeX 参考文献表样式文件    \\
      \bottomrule
    \end{tabular}
    \begin{tablenotes}
      \item [a] 可以通过 xelatex 编译生成模板的使用说明文档;
        使用 xetex 编译 \file{thuthesis.ins} 时则会从 \file{.dtx} 中去除掉文档和注释,得到精简的 \file{.cls} 文件。
      \item [b] 更新模板时,一定要记得编译生成 \file{.cls} 文件,否则编译论文时载入的依然是旧版的模板。
    \end{tablenotes}
  \end{threeparttable}
\end{table}

如某个表需要转页接排,可以使用 \pkg{longtable} 宏包,需要在随后的各页上重复表的编号。
编号后跟表题(可省略)和“(续)”,置于表上方。续表均应重复表头。

\begin{longtable}{cccc}
    \caption{跨页长表格的表题}
    \label{tab:longtable} \\
    \toprule
    表头 1 & 表头 2 & 表头 3 & 表头 4 \\
    \midrule
  \endfirsthead
    \caption*{续表~\thetable\quad 跨页长表格的表题} \\
    \toprule
    表头 1 & 表头 2 & 表头 3 & 表头 4 \\
    \midrule
  \endhead
    \bottomrule
  \endfoot
  Row 1  & & & \\
  Row 2  & & & \\
  Row 3  & & & \\
  Row 4  & & & \\
  Row 5  & & & \\
  Row 6  & & & \\
  Row 7  & & & \\
  Row 8  & & & \\
  Row 9  & & & \\
  Row 10 & & & \\
\end{longtable}



\section{算法}

算法环境可以使用 \pkg{algorithms} 或者 \pkg{algorithm2e} 宏包。

\renewcommand{\algorithmicrequire}{\textbf{输入:}\unskip}
\renewcommand{\algorithmicensure}{\textbf{输出:}\unskip}

\begin{algorithm}
  \caption{Calculate $y = x^n$}
  \label{alg1}
  \small
  \begin{algorithmic}
    \REQUIRE $n \geq 0$
    \ENSURE $y = x^n$

    \STATE $y \leftarrow 1$
    \STATE $X \leftarrow x$
    \STATE $N \leftarrow n$

    \WHILE{$N \neq 0$}
      \IF{$N$ is even}
        \STATE $X \leftarrow X \times X$
        \STATE $N \leftarrow N / 2$
      \ELSE[$N$ is odd]
        \STATE $y \leftarrow y \times X$
        \STATE $N \leftarrow N - 1$
      \ENDIF
    \ENDWHILE
  \end{algorithmic}
\end{algorithm}

% !TeX root = ../thuthesis-example.tex

\chapter{数学符号和公式}

\section{数学符号}

中文论文的数学符号默认遵循 GB/T 3102.11—1993《物理科学和技术中使用的数学符号》
\footnote{原 GB 3102.11—1993,自 2017 年 3 月 23 日起,该标准转为推荐性标准。}。
该标准参照采纳 ISO 31-11:1992 \footnote{目前已更新为 ISO 80000-2:2019。},
但是与 \TeX{} 默认的美国数学学会(AMS)的符号习惯有所区别。
具体地来说主要有以下差异:
\begin{enumerate}
  \item 大写希腊字母默认为斜体,如
    \begin{equation*}
      \Gamma \Delta \Theta \Lambda \Xi \Pi \Sigma \Upsilon \Phi \Psi \Omega.
    \end{equation*}
    注意有限增量符号 $\increment$ 固定使用正体,模板提供了 \cs{increment} 命令。
  \item 小于等于号和大于等于号使用倾斜的字形 $\le$、$\ge$。
  \item 积分号使用正体,比如 $\int$、$\oint$。
  \item
    偏微分符号 $\partial$ 使用正体。
  \item
    省略号 \cs{dots} 按照中文的习惯固定居中,比如
    \begin{equation*}
      1, 2, \dots, n \quad 1 + 2 + \dots + n.
    \end{equation*}
  \item
    实部 $\Re$ 和虚部 $\Im$ 的字体使用罗马体。
\end{enumerate}

以上数学符号样式的差异可以在模板中统一设置。
另外国标还有一些与 AMS 不同的符号使用习惯,需要用户在写作时进行处理:
\begin{enumerate}
  \item 数学常数和特殊函数名用正体,如
    \begin{equation*}
      \uppi = 3.14\dots; \quad
      \symup{i}^2 = -1; \quad
      \symup{e} = \lim_{n \to \infty} \left( 1 + \frac{1}{n} \right)^n.
    \end{equation*}
  \item 微分号使用正体,比如 $\dif y / \dif x$。
  \item 向量、矩阵和张量用粗斜体(\cs{symbf}),如 $\symbf{x}$、$\symbf{\Sigma}$、$\symbfsf{T}$。
  \item 自然对数用 $\ln x$ 不用 $\log x$。
\end{enumerate}


英文论文的数学符号使用 \TeX{} 默认的样式。
如果有必要,也可以通过设置 \verb|math-style| 选择数学符号样式。

关于量和单位推荐使用
\href{http://mirrors.ctan.org/macros/latex/contrib/siunitx/siunitx.pdf}{\pkg{siunitx}}
宏包,
可以方便地处理希腊字母以及数字与单位之间的空白,
比如:
\SI{6.4e6}{m},
\SI{9}{\micro\meter},
\si{kg.m.s^{-1}},
\SIrange{10}{20}{\degreeCelsius}。



\section{数学公式}

数学公式可以使用 \env{equation} 和 \env{equation*} 环境。
注意数学公式的引用应前后带括号,通常使用 \cs{eqref} 命令,比如式\eqref{eq:example}。
\begin{equation}
  \frac{1}{2 \uppi \symup{i}} \int_\gamma f = \sum_{k=1}^m n(\gamma; a_k) \mathscr{R}(f; a_k).
  \label{eq:example}
\end{equation}

多行公式尽可能在“=”处对齐,推荐使用 \env{align} 环境。
\begin{align}
  a & = b + c + d + e \\
    & = f + g
\end{align}



\section{数学定理}

定理环境的格式可以使用 \pkg{amsthm} 或者 \pkg{ntheorem} 宏包配置。
用户在导言区载入这两者之一后,模板会自动配置 \env{thoerem}、\env{proof} 等环境。

\begin{theorem}[Lindeberg--Lévy 中心极限定理]
  设随机变量 $X_1, X_2, \dots, X_n$ 独立同分布, 且具有期望 $\mu$ 和有限的方差 $\sigma^2 \ne 0$,
  记 $\bar{X}_n = \frac{1}{n} \sum_{i+1}^n X_i$,则
  \begin{equation}
    \lim_{n \to \infty} P \left(\frac{\sqrt{n} \left( \bar{X}_n - \mu \right)}{\sigma} \le z \right) = \Phi(z),
  \end{equation}
  其中 $\Phi(z)$ 是标准正态分布的分布函数。
\end{theorem}
\begin{proof}
  Trivial.
\end{proof}

同时模板还提供了 \env{assumption}、\env{definition}、\env{proposition}、
\env{lemma}、\env{theorem}、\env{axiom}、\env{corollary}、\env{exercise}、
\env{example}、\env{remar}、\env{problem}、\env{conjecture} 这些相关的环境。

% !TeX root = ../thuthesis-example.tex

\chapter{引用文献的标注}

模板支持 BibTeX 和 BibLaTeX 两种方式处理参考文献。
下文主要介绍 BibTeX 配合 \pkg{natbib} 宏包的主要使用方法。


\section{顺序编码制}

在顺序编码制下,默认的 \cs{cite} 命令同 \cs{citep} 一样,序号置于方括号中,
引文页码会放在括号外。
统一处引用的连续序号会自动用短横线连接。

\thusetup{
  cite-style = super,
}
\noindent
\begin{tabular}{l@{\quad$\Rightarrow$\quad}l}
  \verb|\cite{zhangkun1994}|               & \cite{zhangkun1994}               \\
  \verb|\citet{zhangkun1994}|              & \citet{zhangkun1994}              \\
  \verb|\citep{zhangkun1994}|              & \citep{zhangkun1994}              \\
  \verb|\cite[42]{zhangkun1994}|           & \cite[42]{zhangkun1994}           \\
  \verb|\cite{zhangkun1994,zhukezhen1973}| & \cite{zhangkun1994,zhukezhen1973} \\
\end{tabular}


也可以取消上标格式,将数字序号作为文字的一部分。
建议全文统一使用相同的格式。

\thusetup{
  cite-style = inline,
}
\noindent
\begin{tabular}{l@{\quad$\Rightarrow$\quad}l}
  \verb|\cite{zhangkun1994}|               & \cite{zhangkun1994}               \\
  \verb|\citet{zhangkun1994}|              & \citet{zhangkun1994}              \\
  \verb|\citep{zhangkun1994}|              & \citep{zhangkun1994}              \\
  \verb|\cite[42]{zhangkun1994}|           & \cite[42]{zhangkun1994}           \\
  \verb|\cite{zhangkun1994,zhukezhen1973}| & \cite{zhangkun1994,zhukezhen1973} \\
\end{tabular}



\section{著者-出版年制}

著者-出版年制下的 \cs{cite} 跟 \cs{citet} 一样。

\thusetup{
  cite-style = author-year,
}
\noindent
\begin{tabular}{@{}l@{$\Rightarrow$}l@{}}
  \verb|\cite{zhangkun1994}|                & \cite{zhangkun1994}                \\
  \verb|\citet{zhangkun1994}|               & \citet{zhangkun1994}               \\
  \verb|\citep{zhangkun1994}|               & \citep{zhangkun1994}               \\
  \verb|\cite[42]{zhangkun1994}|            & \cite[42]{zhangkun1994}            \\
  \verb|\citep{zhangkun1994,zhukezhen1973}| & \citep{zhangkun1994,zhukezhen1973} \\
\end{tabular}

\vskip 2ex
\thusetup{
  cite-style = super,
}
注意,引文参考文献的每条都要在正文中标注
\cite{zhangkun1994,zhukezhen1973,dupont1974bone,zhengkaiqing1987,%
  jiangxizhou1980,jianduju1994,merkt1995rotational,mellinger1996laser,%
  bixon1996dynamics,mahui1995,carlson1981two,taylor1983scanning,%
  taylor1981study,shimizu1983laser,atkinson1982experimental,%
  kusch1975perturbations,guangxi1993,huosini1989guwu,wangfuzhi1865songlun,%
  zhaoyaodong1998xinshidai,biaozhunhua2002tushu,chubanzhuanye2004,%
  who1970factors,peebles2001probability,baishunong1998zhiwu,%
  weinstein1974pathogenic,hanjiren1985lun,dizhi1936dizhi,%
  tushuguan1957tushuguanxue,aaas1883science,fugang2000fengsha,%
  xiaoyu2001chubanye,oclc2000about,scitor2000project%
}。



% 其他部分
\backmatter

% 参考文献
% \bibliography{ref/refs}  % 参考文献使用 BibTeX 编译
% \printbibliography       % 参考文献使用 BibLaTeX 编译

% 附录
% 本科生需要将附录放到声明之后,个人简历之前
\appendix
% % !TeX root = ../thuthesis-example.tex

\begin{survey}
\label{cha:survey}

\title{Title of the Survey}
\maketitle


\tableofcontents


本科生的外文资料调研阅读报告。


\section{Figures and Tables}

\subsection{Figures}

An example figure in appendix (Figure~\ref{fig:appendix-survey-figure}).

\begin{figure}
  \centering
  \includegraphics[width=0.6\linewidth]{example-image-a.pdf}
  \caption{Example figure in appendix}
  \label{fig:appendix-survey-figure}
\end{figure}


\subsection{Tables}

An example table in appendix (Table~\ref{tab:appendix-survey-table}).

\begin{table}
  \centering
  \caption{Example table in appendix}
  \begin{tabular}{ll}
    \toprule
    File name       & Description                                         \\
    \midrule
    thuthesis.dtx   & The source file including documentaion and comments \\
    thuthesis.cls   & The template file                                   \\
    thuthesis-*.bst & BibTeX styles                                       \\
    thuthesis-*.bbx & BibLaTeX styles for bibliographies                  \\
    thuthesis-*.cbx & BibLaTeX styles for citations                       \\
    \bottomrule
  \end{tabular}
  \label{tab:appendix-survey-table}
\end{table}


\section{Equations}

An example equation in appendix (Equation~\eqref{eq:appendix-survey-equation}).
\begin{equation}
  \frac{1}{2 \uppi \symup{i}} \int_\gamma f = \sum_{k=1}^m n(\gamma; a_k) \mathscr{R}(f; a_k)
  \label{eq:appendix-survey-equation}
\end{equation}


\section{Citations}

Example citations in appendix.
\cite{abrahams99tex}
\cite{salomon1995advanced}
\cite{abrahams99tex,salomon1995advanced}


\bibliographystyle{unsrtnat}
\bibliography{ref/appendix}

\end{survey}
       % 本科生:外文资料的调研阅读报告
% % !TeX root = ../thuthesis-example.tex

\begin{translation}
\label{cha:translation}

\title{书面翻译题目}
\maketitle

\tableofcontents


本科生的外文资料书面翻译。


\section{图表示例}

\subsection{图}

附录中的图片示例(图~\ref{fig:appendix-translation-figure})。

\begin{figure}
  \centering
  \includegraphics[width=0.6\linewidth]{example-image-a.pdf}
  \caption{附录中的图片示例}
  \label{fig:appendix-translation-figure}
\end{figure}


\subsection{表格}

附录中的表格示例(表~\ref{tab:appendix-translation-table})。

\begin{table}
  \centering
  \caption{附录中的表格示例}
  \begin{tabular}{ll}
    \toprule
    文件名          & 描述                         \\
    \midrule
    thuthesis.dtx   & 模板的源文件,包括文档和注释 \\
    thuthesis.cls   & 模板文件                     \\
    thuthesis-*.bst & BibTeX 参考文献表样式文件    \\
    thuthesis-*.bbx & BibLaTeX 参考文献表样式文件  \\
    thuthesis-*.cbx & BibLaTeX 引用样式文件        \\
    \bottomrule
  \end{tabular}
  \label{tab:appendix-translation-table}
\end{table}


\section{数学公式}

附录中的数学公式示例(公式\eqref{eq:appendix-translation-equation})。
\begin{equation}
  \frac{1}{2 \uppi \symup{i}} \int_\gamma f = \sum_{k=1}^m n(\gamma; a_k) \mathscr{R}(f; a_k)
  \label{eq:appendix-translation-equation}
\end{equation}


\section{文献引用}

文献引用示例\cite{abrahams99tex}。


\appendix

\section{附录}

附录的内容。


% 书面翻译的参考文献
\bibliographystyle{unsrtnat}
\bibliography{ref/appendix}

% 书面翻译对应的原文索引
\begin{translation-index}
  \nocite{salomon1995advanced}
  \bibliographystyle{unsrtnat}
  \bibliography{ref/appendix}
\end{translation-index}

\end{translation}
  % 本科生:外文资料的书面翻译
% !TeX root = ../thuthesis-example.tex

\chapter{补充内容}

附录是与论文内容密切相关、但编入正文又影响整篇论文编排的条理和逻辑性的资料,例如某些重要的数据表格、计算程序、统计表等,是论文主体的补充内容,可根据需要设置。

附录中的图、表、数学表达式、参考文献等另行编序号,与正文分开,一律用阿拉伯数字编码,
但在数码前冠以附录的序号,例如“图~\ref{fig:appendix-figure}”,
“表~\ref{tab:appendix-table}”,“式\eqref{eq:appendix-equation}”等。


\section{插图}

% 附录中的插图示例(图~\ref{fig:appendix-figure})。

\begin{figure}
  \centering
  \includegraphics[width=0.6\linewidth]{example-image-a.pdf}
  \caption{附录中的图片示例}
  \label{fig:appendix-figure}
\end{figure}


\section{表格}

% 附录中的表格示例(表~\ref{tab:appendix-table})。

\begin{table}
  \centering
  \caption{附录中的表格示例}
  \begin{tabular}{ll}
    \toprule
    文件名          & 描述                         \\
    \midrule
    thuthesis.dtx   & 模板的源文件,包括文档和注释 \\
    thuthesis.cls   & 模板文件                     \\
    thuthesis-*.bst & BibTeX 参考文献表样式文件    \\
    thuthesis-*.bbx & BibLaTeX 参考文献表样式文件  \\
    thuthesis-*.cbx & BibLaTeX 引用样式文件        \\
    \bottomrule
  \end{tabular}
  \label{tab:appendix-table}
\end{table}


\section{数学表达式}

% 附录中的数学表达式示例(式\eqref{eq:appendix-equation})。
\begin{equation}
  \frac{1}{2 \uppi \symup{i}} \int_\gamma f = \sum_{k=1}^m n(\gamma; a_k) \mathscr{R}(f; a_k)
  \label{eq:appendix-equation}
\end{equation}


\section{参考文献}

附录中的参考文献示例(\cite{carlson1981two} 和 \cite{carlson1981two,taylor1983scanning,taylor1981study})。

\printbibliography


% 致谢
% !TeX root = ../thuthesis-example.tex

\begin{acknowledgements}
  衷心感谢导师×××教授和物理系××副教授对本人的精心指导。他们的言传身教将使我终生受益。

  在美国麻省理工学院化学系进行九个月的合作研究期间,承蒙 Robert Field 教授热心指导与帮助,不胜感激。

  感谢×××××实验室主任×××教授,以及实验室全体老师和同窗们学的热情帮助和支持!

  本课题承蒙国家自然科学基金资助,特此致谢。
\end{acknowledgements}


% 声明
\statement
% 将签字扫描后的声明文件 scan-statement.pdf 替换原始页面
% \statement[file=scan-statement.pdf]
% 本科生编译生成的声明页默认不加页脚,插入扫描版时再补上;
% 研究生编译生成时有页眉页脚,插入扫描版时不再重复。
% 也可以手动控制是否加页眉页脚
% \statement[page-style=empty]
% \statement[file=scan-statement.pdf, page-style=plain]

% 个人简历、在学期间完成的相关学术成果
% 本科生可以附个人简历,也可以不附个人简历
% !TeX root = ../thuthesis-example.tex

\begin{resume}

  \section*{个人简历}

  197× 年 ×× 月 ×× 日出生于四川××县。

  1992 年 9 月考入××大学化学系××化学专业,1996 年 7 月本科毕业并获得理学学士学位。

  1996 年 9 月免试进入清华大学化学系攻读××化学博士至今。


  \section*{在学期间完成的相关学术成果}

  \subsection*{学术论文}

  \begin{achievements}
    \item Yang Y, Ren T L, Zhang L T, et al. Miniature microphone with silicon-based ferroelectric thin films[J]. Integrated Ferroelectrics, 2003, 52:229-235.
    \item 杨轶, 张宁欣, 任天令, 等. 硅基铁电微声学器件中薄膜残余应力的研究[J]. 中国机械工程, 2005, 16(14):1289-1291.
    \item 杨轶, 张宁欣, 任天令, 等. 集成铁电器件中的关键工艺研究[J]. 仪器仪表学报, 2003, 24(S4):192-193.
    \item Yang Y, Ren T L, Zhu Y P, et al. PMUTs for handwriting recognition. In press[J]. (已被Integrated Ferroelectrics录用)
  \end{achievements}


  \subsection*{专利}

  \begin{achievements}
    \item 任天令, 杨轶, 朱一平, 等. 硅基铁电微声学传感器畴极化区域控制和电极连接的方法: 中国, CN1602118A[P]. 2005-03-30.
    \item Ren T L, Yang Y, Zhu Y P, et al. Piezoelectric micro acoustic sensor based on ferroelectric materials: USA, No.11/215, 102[P]. (美国发明专利申请号.)
  \end{achievements}

\end{resume}


% 指导教师/指导小组评语
% 本科生不需要
% !TeX root = ../thuthesis-example.tex

\begin{comments}
% \begin{comments}[name = {指导小组评语}]
% \begin{comments}[name = {Comments from Thesis Supervisor}]
% \begin{comments}[name = {Comments from Thesis Supervision Committee}]

  论文提出了……

\end{comments}


% 答辩委员会决议书
% 本科生不需要
% !TeX root = ../thuthesis-example.tex

\begin{resolution}

  论文提出了……

  论文取得的主要创新性成果包括:

  1. ……

  2. ……

  3. ……

  论文工作表明作者在×××××具有×××××知识,具有××××能力,论文××××,答辩××××。

  答辩委员会表决,(×票/一致)同意通过论文答辩,并建议授予×××(姓名)×××(门类)学博士/硕士学位。

\end{resolution}


% 本科生的综合论文训练记录表(扫描版)
% \record{file=scan-record.pdf}

\end{document}
