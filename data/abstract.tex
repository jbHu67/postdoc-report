% !TeX root = ../thuthesis-example.tex

% 中英文摘要和关键字

\begin{abstract}
  拓扑优化利用计算手段优化材料分布,旨在获得预期性能并满足特定约束的结构,是现代智能制造领域的重要设计方法。然而,拓扑优化过程中产生的结构通常具有非常复杂的几何与拓扑特征,给结构的表示、分析和优化都带来了极大的挑战。本工作利用隐式表示灵活性和可微分性的特点,针对不同的结构设计目标和约束,以提高算法效率和应用可扩展性为目的,开展了以下创新性研究工作:
  
(1)为了解决拓扑优化分析和迭代周期长的问题,

  % 关键词用“英文逗号”分隔,输出时会自动处理为正确的分隔符
  \thusetup{
    keywords = {拓扑优化, 隐式表示, 变分自编码器, 薄壳结构, 自支撑结构},
  }
\end{abstract}

\begin{abstract*}
  An abstract of a dissertation is a summary and extraction of research work and contributions.
  Included in an abstract should be description of research topic and research objective, brief introduction to methodology and research process, and summary of conclusion and contributions of the research.
  An abstract should be characterized by independence and clarity and carry identical information with the dissertation.
  It should be such that the general idea and major contributions of the dissertation are conveyed without reading the dissertation.

  An abstract should be concise and to the point.
  It is a misunderstanding to make an abstract an outline of the dissertation and words “the first chapter”, “the second chapter” and the like should be avoided in the abstract.

  Keywords are terms used in a dissertation for indexing, reflecting core information of the dissertation.
  An abstract may contain a maximum of 5 keywords, with semi-colons used in between to separate one another.

  % Use comma as separator when inputting
  \thusetup{
    keywords* = {keyword 1, keyword 2, keyword 3, keyword 4, keyword 5},
  }
\end{abstract*}
