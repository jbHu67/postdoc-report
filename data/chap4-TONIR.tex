% !TeX root = ../thuthesis-example.tex

\chapter{基于隐式神经表示的端到端拓扑优化方法}

\section{引言}
拓扑优化(Topology Optimization,TO)是一种计算设计方法论,旨在确定设计域内材料的最佳分布,以实现特定的性能目标,同时遵守给定的约束条件~\cite{sigmund2013}。在过去的三十年里,计算机技术和3D打印的进步推动了拓扑优化方法的重大进展。这些方法包括固体各向同性材料惩罚(SIMP)方法~\cite{bendsoe1999}、进化结构优化(ESO)方法~\cite{xie1993}、水平集方法(LSM)\cite{wang2003level}以及最近的可移动可变组件/空隙(MMC/MMV)方法\cite{guo2014doing,zhang2017explicit}。这些方法在结构表示方式上有所不同,但它们都是通过重复的物理响应分析和参数梯度信息来迭代地寻求最优的材料分布。拓扑优化的求解过程涉及反复的、计算密集型的有限元分析,以求解物理平衡方程。这种计算需求是这些方法效率低下的主要原因,从而阻碍了拓扑优化在实际应用场景中的推广。