\chapter{基于隐式表示的薄壳结构雕刻设计}

\section{引言}
薄壳结构以其纤薄和弯曲的形状承载载荷,这种结构特别优雅和高效。它们在我们的生活环境中以各种尺寸广泛存在,在满足力学性能的同时提供了艺术美感的视觉体验~\cite{adriaenssens2014,melaragno2012}。近年来,基于壳体结构的曲面设计受到艺术家和研究人员的广泛关注~\cite{Pietroni2015, Chen2016, Liu2020}。其中,雕刻是设计壳体结构的一种流行方式~\cite{Yang2019,stadlbauer2020}。经过精巧的雕刻设计,壳体结构可以具有更高的艺术感,并在医疗和轻量化应用中得到广泛实践~\cite{Zhang2017,Rao2019}。

现代计算机图形学技术的发展已经引领了壳体结构雕刻设计方法的日益多样化,例如基于纹理合成的方法~\cite{Dumas2015},基于镶嵌的方法~\cite{Pietroni2015}和基于重复模板的方法~\cite{schumacher2016}。作为一种材料减少过程,雕刻设计的核心问题是在设计过程中维持壳体结构的力学性能和功能。现有的大多数方法都侧重于显式表示,如多边形网格等,这不利于结构分析和参数优化。
由于雕刻壳体结构的拓扑和几何结构高度复杂,采用传统有限元方法(FEM)~\cite{bucalem1997,cirak2002} 进行力学响应分析非常耗时。此外,在现有的设计技术中,设计、分析和优化通常是分离的,在不同阶段经常需要重复重新划分网格~\cite{panetta2019}。由于缺乏统一的表示方法和有效的优化技术,在壳体结构上进行雕刻设计并非易事。

本文研究提出了一种隐式参数化方法来设计轻量级薄壳结构,并通过参数优化保证结构的物理可靠性。具体地,通过在输入薄壳结构上分布重复模板图案并对其进行雕刻来完成设计,用隐式表示的薄壳结构的模板图案可以直接用函数进行设计、分析和优化,本文通过优化模板图案的尺寸和方向等属性,在雕刻设计的同时最大化壳体结构的刚度。与基于传统有限元的网格方法相比,本文方法由于避免了显式模型的生成和分析过程中的重新网格剖分,在保证足够精度的前提下大大提高了计算效率。此外,本文还通过对图案模板的不同函数操作,实现了对偶雕刻设计,进一步增强了薄壳结构设计的丰富性。本文通过在多种薄壳结构模型上的测试,并实施了仿真和对比试验来证明该方法的有效性和高效性。

\section{研究方法}
